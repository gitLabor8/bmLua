\documentclass[10pt]{article}
\usepackage[text={15cm,21.0cm},centering]{geometry}
\usepackage{hyperref}
\usepackage{prooftree}
\usepackage{alltt}
\usepackage{stmaryrd}
\usepackage{multicol}
\usepackage{varwidth}
\usepackage{color}
\usepackage{afterpage}
\usepackage[figuresleft]{rotating}
\newcommand{\calA}[1]{{\cal A}\llbracket #1 \rrbracket}
\newcommand{\calB}[1]{{\cal B}\llbracket #1 \rrbracket}
\newcommand{\CC}{{\mathbb C}}
\newcommand{\desda}{\leftrightarrow}
\newcommand{\Desda}{\Leftrightarrow}
\newcommand{\dimpl}{\Rightarrow}
\newcommand{\espace}{\hspace*{0.2cm}}
\newcommand{\impl}{\rightarrow}
\newcommand{\mgeq}{\begin{math}\geq\end{math}}
\newcommand{\mggd}{\mbox{ggd}}
\newcommand{\mgcd}{\mbox{gcd}}
\newcommand{\minv}{\mbox{INV}}
\newcommand{\mneg}{\begin{math}\neg\end{math}}
\newcommand{\mneq}{\begin{math}\neq\end{math}}
\newcommand{\mwedge}{\begin{math}\wedge\end{math}}
\newcommand{\NN}{{\mathbb N}}
\newcommand{\postboven}{}
\newcommand{\postonder}{}
\newcommand{\preboven}{}
\newcommand{\preonder}{}
\newcommand{\punten}[1]{\marginpar{\fbox{#1}}}
\newcommand{\QQ}{{\mathbb Q}}
\newcommand{\RR}{{\mathbb R}}

\newcommand{\pass}[2]{\left[#1\mapsto\calA{#2}\right]}
\newcommand{\psq}[4]{\langle#1,#2\rangle \Longrightarrow \langle#3,#4\rangle}
\newcommand{\pss}[3]{\langle#1,#2\rangle \Longrightarrow #3}
\newcommand{\psqstar}[4]{\langle#1,#2\rangle \Longrightarrow^* \langle#3,#4\rangle}



\newcommand{\psqsr}[5]{\{#1\}\espace #2 \espace \{#3\} \espace #4 \espace \{#5\}}
\newcommand{\psqsrst}[7]{\{#1\}\espace #2 \espace \{#3\} \espace #4 \espace \{#5\} \espace #6 \espace \{#7\}}
\newcommand{\psqan}[5]{\{#1\}^{#4}\espace #2 \espace \{#3\}^{#5}}
\newcommand{\allttmath}[1]{\begin{math}#1\end{math}}
\newcommand{\allttsup}[1]{\begin{math}^{#1}\end{math}}
\newcommand{\allttsub}[1]{\begin{math}_{#1}\end{math}}

\newcommand{\sblock}[2]{\mbox{\texttt{begin~}}#1\espace#2\mbox{\texttt{~end}}}
\newcommand{\svar}[2]{\mbox{\texttt{var~}}#1:=#2;}
\newcommand{\sass}[2]{#1:=#2}
\newcommand{\scomp}[2]{#1;#2}
\newcommand{\sifthenelse}[3]{\mbox{\texttt{if~}}#1\mbox{\texttt{~then~}}#2\mbox{\texttt{~else~}}#3}
\newcommand{\sor}[2]{#1\mbox{\texttt{~or~}}#2}
\newcommand{\srepeatuntil}[2]{\mbox{\texttt{repeat~}}#1\mbox{\texttt{~until~}}#2}
\newcommand{\sskip}{\mbox{\texttt{skip}}}
\newcommand{\swhile}[2]{\mbox{\texttt{while~}}#1\mbox{\texttt{~do~}}#2}
\newcommand{\swhileuntil}[3]{\mbox{\texttt{while~}}#1\mbox{\texttt{~do~}}#2\mbox{\texttt{~until~}}#3}
\newcommand{\trans}[3]{\langle #1, #2\rangle \impl #3}
\newcommand{\ZZ}{{\mathbb Z}}
\newenvironment{sol}{\expandafter\ifsols\par{\bf{Solution:}\\}}{\par\expandafter\fi}
\newcommand{\mtt}{\mbox{tt}}
\newcommand{\mff}{\mbox{ff}}
\newcommand{\mwhile}{\mbox{while}}
\newcommand{\mif}{\mbox{if}}
\newcommand{\mthen}{\mbox{then}}
\newcommand{\melse}{\mbox{else}}
\newcommand{\mdo}{\mbox{do}}
\newcommand{\mmskip}{\mbox{skip}}
\newcommand{\mtrue}{\mbox{true}}
\newcommand{\mfalse}{\mbox{false}}
\newcommand{\rskip}{[\mbox{skip}]}
\newcommand{\rass}{[\mbox{ass}]}
\newcommand{\rcomp}{[\mbox{comp}]}
\newcommand{\rif}{[\mbox{if}]}
\newcommand{\rwhile}{[\mbox{while}]}
\newcommand{\rcons}{[\mbox{cons}]}
\newcommand{\rvar}{[\mbox{var}]}
\newcommand{\rnone}{[\mbox{none}]}
\newcommand{\riftt}{[\mbox{if}_{\mbox{tt}}]}
\newcommand{\rifff}{[\mbox{if}_{\mbox{ff}}]}
\newcommand{\rwhilett}{[\mbox{while}_{\mbox{tt}}]}
\newcommand{\rwhileff}{[\mbox{while}_{\mbox{ff}}]}
\newcommand{\rwhileuntilttff}{[\mbox{while-until}^{\mbox{ttff}}]}
\newcommand{\rwhileuntiltttt}{[\mbox{while-until}^{\mbox{tttt}}]}
\newcommand{\rwhileuntilff}{[\mbox{while-until}^{\mbox{ff}}]}
\newcommand{\rcall}{[\mbox{call}]}
\newcommand{\rcallrec}{[\mbox{call}^{\mbox{rec}}]}
\newcommand{\strans}[4]{{#1} \vdash \langle {#2}, {#3} \rangle \rightarrow {#4}}
\newcommand{\dtrans}[3]{\langle {#1}, {#2} \rangle \rightarrow_{D} {#3}}

\newcommand{\axjustifies}{\thickness0em\justifies}
\endinput


\newcommand{\pass}[2]{\left[#1\mapsto\calA{#2}\right]}
\newcommand{\smaller}[1]{{\small\textcolor{red}{#1}}}
\renewcommand{\allttsup}[1]{\begin{math}^{\textcolor{red}{#1}}\end{math}}
%\renewcommand{\allttsub}[1]{\begin{math}_{\textcolor{red}{#1}}\end{math}}
\renewcommand{\psqan}[5]{\{#1\}^{\textcolor{red}{#4}}\espace #2 \espace \{#3\}^{\textcolor{red}{#5}}}

\title{Annotated Programs}

\author{Engelbert Hubbers}

\begin{document}
\maketitle
\section*{Introduction}
In chapter 6 of their book \cite{nie}, Nielson and Nielson introduce
the system of axiomatic program verification.
The rules for this system are given as axioms and derivation rules in
Table~6.1 on page~178.
In particular this means that proofs in this system are always given as
derivation trees.
Unfortunately these trees can get very wide and the general proof is
a bit hard to follow: you basically start with a precondition at the
bottom of the tree, work through all smaller trees in a depth first scenario
to finally end with the desired post-condition again at the bottom of the tree.
In order to improve the readability of the proofs, we introduce a
different notation that allows us to write the correctness proofs in a
short and comprehensible way.

\section{Intuitive annotations}
The idea is that we write the main precondition above the first line of code,
all intermediate predicates between the code, and the final post-condition below
the last line of code.
Let's give an example.

Consider the program $\swhile{\mtrue}{\sskip}$.
We would like to prove here\footnote{Note that in \cite{nie} a different assertion is proved: $\psq{\mtrue}{\swhile{\mtrue}{\sskip}}{\mtrue}$. Convince yourself that both assertions are indeed correct.} that the assertion
\[\psq{\mtrue}{\swhile{\mtrue}{\sskip}}{\mfalse}\]
holds
Figure~\ref{fig:treestyle} shows the proof in terms of trees as 
defined in \cite{nie}.
\begin{figure}[htb]
$$
\begin{prooftree}
\[
  \[
    \[
      \axjustifies
      \psqan{\mtrue}{\sskip}{\mtrue}{4}{5}
      \using{\rassp}
    \]
    \justifies
    \psqan{\mtrue \land \mtrue}{\sskip}{\mtrue}{3}{6}
    \using{\rconsp}
  \]
  \justifies
  \psqan{\mtrue}{\swhile{\mtrue}{\sskip}}{\neg\mtrue \land \mtrue}{2}{7}
  \using{\rwhilep}
\]
\justifies
\psqan{\mtrue}{\swhile{\mtrue}{\sskip}}{\mfalse}{1}{8}
\using{\rconsp}
\end{prooftree}
$$
\caption{Proof in normal tree style}
\label{fig:treestyle}
\end{figure}
Note that the numbers attached to the predicates are officially
not part of this
tree style and therefore they are displayed in a
different color.
They are here only to match a specific predicate in this tree
to a specific predicate in the annotated program style.

Figure~\ref{fig:annostyle} shows two versions of the proof in our annotated
program style.

\begin{figure}[htb]
\begin{center}
\begin{multicols}{2}
\begin{varwidth}{3cm}
\begin{alltt}
\{\mtrue\}\allttsup{1}
\{\mtrue\}\allttsup{2}
\mwhile \mtrue
\mdo
  \{\mtrue \allttmath{\land} \mtrue\}\allttsup{3}
  \{\mtrue\}\allttsup{4}
  \mmskip
  \{\mtrue\}\allttsup{5}
  \{\mtrue\}\allttsup{6}
\{\allttmath{\neg}\mtrue \allttmath{\land} \mtrue\}\allttsup{7}
\{\mfalse\}\allttsup{8}
\end{alltt}
\end{varwidth}
\newpage
\begin{varwidth}{3cm}
\begin{alltt}
\{\mtrue\}\allttsup{1}
\mwhile \mtrue
\mdo
  \{\mtrue \allttmath{\land} \mtrue\}\allttsup{3}
  \{\mtrue\}\allttsup{4}
  \mmskip
  \{\mtrue\}\allttsup{5}
\{\allttmath{\neg}\mtrue \allttmath{\land} \mtrue\}\allttsup{7}
\{\mfalse\}\allttsup{8}
\end{alltt}
\end{varwidth}
\end{multicols}
\end{center}
\caption{Proofs in annotated program style}
\label{fig:annostyle}
\end{figure}

But how should we read this?
Let's have a look at the proof on the left.
From top to bottom, each time we encounter another predicate, it should
follow from the one directly above.
If there are no statements between these predicates, the lower
predicate should be directly implied by the one above as is for instance the
case for predicates 7 and 8.
And if there are statements between the predicates
as is the case with predicates 4 and 5, it means that
the one above the statement
acts as a precondition to this statement, and the one below
acts as a post-condition to it.
As a consequence, the numbering of the predicates is now in a normal
order. At the first line the initial precondition is written and at the last
line the desired post-condition is written.

Note that this numbering is not part of the official style.
As stated before, it is only used to indicate which predicate in the program
corresponds to which predicate in the tree.
In particular, the numbers help us to recognize the rules being used.
Predicates 1, 2, 7 and 8 correspond with the lower $\rconsp$ rule in the tree.
Predicates 2, 3, 6 and 7 correspond with the $\rwhilep$ rule.
Predicates 3, 4, 5 and 6 correspond with the upper $\rconsp$ rule.
Predicates 4 and 5 correspond with the $\rskipp$ axiom.

Now if we take a look at the proof on the right, we see that the predicates 2 
and 6 have disappeared.
This is because they are exactly the same as the predicates 1 and 5 directly
above them.
In general a $\rconsp$ rule deals with four predicates.
However, in case we are only weakening the postcondition, but keep the same
precondition like we did in the lower $\rconsp$ rule, we do not copy the
precondition, but simply write it down once. And likewise if we are only
weakening the precondition as we did with the upper $\rconsp$ rule.
The result is the proof on the right.

\section{Formalization}
Now that we have explained the idea behind the notation, we can introduce
it formally by giving the translation for all axioms and rules.
See Table~\ref{tab:transform} for the details.
%\afterpage{clearpage}
\begin{table}[p]
\caption{Transformation between tree style presentation and annotated program style}
\label{tab:transform}
\[
\begin{array}{|l|l|l|}
\hline
\mbox{Rule} & \mbox{Tree style} & \mbox{Program style}\\ \hline 
\rassp
&
\psq{P\pass{x}{a}}{\sass{x}{a}}{P}
&
\begin{minipage}{1in}
\begin{alltt}

\{\allttmath{P\pass{x}{a}}\}
\sass{x}{a}
\{\allttmath{P}\}

\end{alltt}
\end{minipage}
\\
\hline
\rskipp
&
\psq{P}{\sskip}{P}
&
\begin{minipage}{1in}
\begin{alltt}

\{\allttmath{P}\}
\mmskip
\{\allttmath{P}\}

\end{alltt}
\end{minipage}
\\
\hline
\rcompp
&
$$
\begin{prooftree}
\psq{P}{S_1}{Q}
\quad
\psq{Q}{S_2}{R}
\justifies
\psq{P}{\scomp{S_1}{S_2}}{R}
\end{prooftree}
$$
&
\begin{minipage}{1in}
\begin{alltt}

\{\allttmath{P}\}
\allttmath{S}\allttsub{1}
\{\allttmath{Q}\}
\allttmath{S}\allttsub{2}
\{\allttmath{R}\}

\end{alltt}
\end{minipage}
\\
\hline
\rifp
&
\begin{prooftree}
\psq{\calB{b} \land P}{S_1}{Q}
\quad
\psq{\neg\calB{b} \land P}{S_2}{Q}
\justifies
\psq{P}{\sifthenelse{b}{S_1}{S_2}}{Q}
\end{prooftree}
&
\begin{minipage}{1in}
\begin{alltt}

\{\allttmath{P}\}
\mif \allttmath{b}
\mthen
  \{\allttmath{\calB{b} \land P}\}
  \allttmath{S}\allttsub{1}
  \{\allttmath{Q}\}
\melse
  \{\allttmath{\neg\calB{b} \land P}\}
  \allttmath{S}\allttsub{2}
  \{\allttmath{Q}\}
\{\allttmath{Q}\}

\end{alltt}
\end{minipage}
\\ 
\hline
\rwhilep
&
\begin{prooftree}
\psq{\calB{b} \land P}{S}{P}
\justifies
\psq{P}{\swhile{b}{S}}{\neg\calB{b} \land P}
\end{prooftree}
&
\begin{minipage}{1in}
\begin{alltt}

\{\allttmath{P}\}
\mwhile \allttmath{b}
\mdo
  \{\allttmath{\calB{b} \land P}\}
  \allttmath{S}
  \{\allttmath{P}\}
\{\allttmath{\neg\calB{b} \land P}\}

\end{alltt}
\end{minipage}
\\ 
\hline
\rconsp
&
\begin{prooftree}
\psq{P'}{S}{Q'}
\justifies
\psq{P}{S}{Q}
\end{prooftree}
\quad
\mbox{if }P\Rightarrow P' \mbox{ and } Q' \Rightarrow Q
&
\begin{minipage}{1in}
\begin{alltt}

\{\allttmath{P}\}
\{\allttmath{P'}\}
\allttmath{S}
\{\allttmath{Q'}\}
\{\allttmath{Q}\}

\end{alltt}
\end{minipage}
\\
\hline
\end{array}
\]
\end{table}


In addition to this table we have to make two remarks.
\begin{enumerate}
\item
As we have seen before the general $\rconsp$ rule deals with four predicates,
but in case 
$P=P'$ or $Q=Q'$, the predicate will be listed only once.
\item
Since there is no $\rcompp$ rule for a composition of three or more
statements,
the derivation tree style automatically treats
$\scomp{S_1}{\scomp{S_2}{S_3}}$
as a nested composition of 
$\scomp{S_1}{(\scomp{S_2}{S_3})}$ or
$\scomp{(\scomp{S_1}{S_2})}{S_3}$.
Table~\ref{tab:nestedcomp} shows how we deal with this nesting in the 
annotated program style.
As you can see, again we opt for eliminating duplicates as much as possible.
\begin{table}[t]
\caption{Nested compositions}
\label{tab:nestedcomp}
\[
\begin{array}{|l|l|}\hline
\mbox{Tree style} & \mbox{Program style} \\
\hline
$$
\begin{prooftree}
\psq{P}{S_1}{Q}
\quad
\[
  \psq{Q}{S_2}{R}
  \quad
  \psq{R}{S_3}{T}
  \justifies
  \psq{Q}{\scomp{S_2}{S_3}}{T}
  \using{\rcompp}
\]
\justifies
\psq{P}{\scomp{S_1}{\scomp{S_2}{S_3}}}{T}
\using{\rcompp}
\end{prooftree}
$$
&
\begin{minipage}{1in}
\begin{alltt}

\{\allttmath{P}\}
\allttmath{S}\allttsub{1}
\{\allttmath{Q}\}
\allttmath{S}\allttsub{2}
\{\allttmath{R}\}
\allttmath{S}\allttsub{3}
\{\allttmath{T}\}

\end{alltt}
\end{minipage}
\\
\hline
\end{array}
\]
\end{table}
\end{enumerate}

\section{Example}
Figure~\ref{fig:euclid} shows a 
program which is known as Euclid's algorithm.
\begin{figure}[hb]
\begin{center}
\begin{varwidth}{3cm}
\begin{alltt}
x:=m;
y:=n;
while x{\mneq}y
do
  if x>y
  then
    x:=x-y;
  else
    y:=y-x;
g:=x;
\end{alltt}
\end{varwidth}
\end{center}
\caption{Euclid's algorithm}
\label{fig:euclid}
\end{figure}
We want to prove that this algorithm computes the greatest common divisor
of the values $m$ and $n$.
Or in other words, we want to prove 
\begin{equation}
\label{eqn:euclidgoal}
\psq{m>0 \land n>0}{S}{g=\gcd(m,n)}
\end{equation}
where $S$ is the whole program.
In order to prove this we may assume that
\begin{equation}
\label{eqn:euclidhint}
u>v>0 \mbox{ implies } \gcd(u,v)=\gcd(u-v,v)
\end{equation}

\newcommand{\sasseen}{\sass{x}{m}}
\newcommand{\sasstwee}{\sass{y}{n}}
\newcommand{\sassdrie}{\sass{x}{x-y}}
\newcommand{\sassvier}{\sass{y}{y-x}}
\newcommand{\sassvijf}{\sass{g}{x}}
\newcommand{\sifcond}{x>y}
\newcommand{\sifeen}{\sifthenelse{\sifcond}{\sassdrie}{\sassvier}}
\newcommand{\swhilecond}{x\neq y}
\newcommand{\swhilestat}{\sifeen}
\newcommand{\swhileeen}{\swhile{\swhilecond}{(\swhilestat)}}
\newcommand{\scompeen}{\scomp{\sasseen}{\scomptwee}}
\newcommand{\scomptwee}{\scomp{\sasstwee}{\scompdrie}}
\newcommand{\scompdrie}{\scomp{\swhileeen}{\sassvijf}}
\newcommand{\peena}{m>0}
\newcommand{\peenb}{n>0}
%\newcommand{\peen}{m>0 \land n>0}
\newcommand{\peen}{\peena \land \peenb}
\newcommand{\ptweea}{\gcd(m,n)=\gcd(m,n)}
%\newcommand{\ptwee}{m>0 \land n>0 \land m>0 \land n>0 \land \gcd(m,n)=\gcd(m,n)}
\newcommand{\ptwee}{\peen \land \peena \land \peenb \land \ptweea}
\newcommand{\pdriea}{x>0}
\newcommand{\pdrieb}{\gcd(x,n)=\gcd(m,n)}
%\newcommand{\pdrie}{m>0 \land n>0 \land x>0 \land n>0 \land \gcd(x,n)=\gcd(m,n)}
\newcommand{\pdrie}{\peen \land \pdriea \land \peenb \land \pdrieb}
\newcommand{\pviera}{y>0}
\newcommand{\pvierb}{\gcd(x,y)=\gcd(m,n)}
%\newcommand{\pvier}{m>0 \land n>0 \land x>0 \land y>0 \land \gcd(x,y)=\gcd(m,n)}
\newcommand{\pvier}{\peen \land \pdriea \land \pviera \land \pvierb}
\newcommand{\pvijfa}{x\neq y}
\newcommand{\pvijf}{\pvijfa \land \pvier}
\newcommand{\pzesa}{x>y}
\newcommand{\pzes}{\pzesa \land \pvijf}
\newcommand{\pzevena}{x-y>0}
\newcommand{\pzevenb}{\gcd(x-y,y)=\gcd(m,n)}
%\newcommand{\pzeven}{m>0 \land n>0 \land x-y>0 \land y>0 \land \gcd(x-y,y)=\gcd(m,n)}
\newcommand{\pzeven}{\peen \land \pzevena \land \pviera \land \pzevenb}
\newcommand{\pacht}{\neg(\pzesa) \land \pvijf}
\newcommand{\pnegena}{y>x}
%\newcommand{\pnegen}{y>x \land m>0 \land n>0 \land x>0 \land y>0 \land \gcd(x,y)=\gcd(m,n)}
\newcommand{\pnegen}{\pnegena \land \pvier}
\newcommand{\ptiena}{y-x>0}
\newcommand{\ptienb}{\gcd(x,y-x)=\gcd(m,n)}
%\newcommand{\ptien}{m>0 \land n>0 \land x>0 \land y-x>0 \land \gcd(x,y-x)=\gcd(m,n)}
\newcommand{\ptien}{\peen \land \pdriea \land \ptiena \land \ptienb}
\newcommand{\pelf}{\neg(\pvijfa) \land \pvier}
\newcommand{\ptwaalfa}{x=y}
%\newcommand{\ptwaalf}{x=y \land m>0 \land n>0 \land x>0 \land y>0 \land \gcd(x,y)=\gcd(m,n)}
\newcommand{\ptwaalf}{\ptwaalfa \land \pvier}
\newcommand{\pdertiena}{x=\gcd(m,n)}
%\newcommand{\pdertien}{m>0 \land n>0 \land x=\gcd(m,n)}
\newcommand{\pdertien}{\peen \land \pdertiena}
\newcommand{\pveertiena}{g=\gcd(m,n)}
%\newcommand{\pveertien}{m>0 \land n>0 \land g=\gcd(m,n)}
\newcommand{\pveertien}{\peen \land \pveertiena}
\newcommand{\pvijftien}{\pveertiena}
\newcommand{\speen}{P_{1}}
\newcommand{\sptwee}{P_{2}}
\newcommand{\spdrie}{P_{3}}
\newcommand{\spvier}{P_{4}}
\newcommand{\spvijf}{P_{5}}
\newcommand{\spzes}{P_{6}}
\newcommand{\spzeven}{P_{7}}
\newcommand{\spacht}{P_{8}}
\newcommand{\spnegen}{P_{9}}
\newcommand{\sptien}{P_{10}}
\newcommand{\spelf}{P_{11}}
\newcommand{\sptwaalf}{P_{12}}
\newcommand{\spdertien}{P_{13}}
\newcommand{\spveertien}{P_{14}}
\newcommand{\spvijftien}{P_{15}}
Figure~\ref{fig:euclidtree} gives a proof in tree style.
It automatically shows the problem with this type of proofs:
even if we abbreviate all predicates, it is still difficult to print
the whole tree on one sheet of A4-paper!
\begin{sidewaysfigure}
\vspace*{3cm}
{\small
$$
\hspace*{-2cm}
\begin{prooftree}
\[
  \[
    \axjustifies
    \psq{\sptwee}{\sasseen}{\spdrie}
    \using{\rassp}
  \]
  \quad
  \[
    \[
      \axjustifies
      \psq{\spdrie}{\sasstwee}{\spvier}
      \using{\rassp}
    \]
    \quad
    \[
      \[
        \[
          \[
            \[
              \[
                \[
                  \axjustifies
                  \psq{\spzeven}{\sassdrie}{\spvier}
                  \using{\rassp}
                \]
                \justifies
                \psq{\spzes}{\sassdrie}{\spvier}
                \using{\rconsp}
              \]
              \quad
              \[
                \[
                  \[
                    \axjustifies
                    \psq{\sptien}
{\sassvier}{\spvier}
                    \using{\rassp}
                  \]
                  \justifies
                  \psq{\spnegen}{\sassvier}{\spvier}
                  \using{\rconsp}
                \]
                \justifies
                \psq{\spacht}{\sassvier}{\spvier}
                \using{\rconsp}
              \]
              \justifies
              \psq{\spvijf}{\sifeen}{\spvier}
              \using{\rifp}
            \]
            \justifies
            \psq{\spvier}{\swhileeen}{\spelf}
            \using{\rwhilep}
          \]
          \justifies
          \psq{\spvier}{\swhileeen}{\sptwaalf}
          \using{\rconsp}
        \]
        \justifies
        \psq{\spvier}{\swhileeen}{\spdertien}
        \using{\rconsp}
      \]
      \quad
      \[
        \axjustifies
        \psq {\spdertien} {\sassvijf}{\spveertien}
        \using{\rassp}
      \]
      \justifies
      \psq{\spvier}{\scompdrie}{\spveertien}
      \using{\rcompp}
    \]
    \justifies
    \psq{\spdrie}{\scomptwee}{\spveertien}
    \using{\rcompp}
  \]
  \justifies
  \psq{\sptwee}{\scompeen}{\spveertien}
  \using{\rcompp}
\]
\justifies
\psq{\speen}{\scompeen}{\spvijftien}
\using{\rconsp}
\end{prooftree}
$$
}
\caption{Euclid's algorithm proved by a tree}
\label{fig:euclidtree}
\end{sidewaysfigure}
Before we will go into the details of the proof itself,
we present the same proof in annotated program style
in Figure~\ref{fig:euclidprog}.
The first thing that strikes is that the representation
of this proof is short enough to allow us to include some extra information:
\begin{itemize}
\item
Line numbers that we can use later on to explain the individual steps in the 
proof
\item
The abbreviations used in Figure~\ref{fig:euclidtree}.
\end{itemize}
As before, we use a different color for this extra information, which is not
officially part of the style.
\begin{figure}[htb]
\begin{center}
\begin{varwidth}{15cm}
\begin{alltt}
\smaller{ 01.} \allttmath{\{\peen\}} \smaller{ \allttmath{= \{\speen\}}}
\smaller{ 02.} \allttmath{\{\ptwee\}} \smaller{ \allttmath{= \{\sptwee\}}}
\smaller{    } x:=m;
\smaller{ 03.} \allttmath{\{\pdrie\}} \smaller{ \allttmath{= \{\spdrie\}}}
\smaller{    } y:=n;
\smaller{ 04.} \allttmath{\{\pvier\}} \smaller{ \allttmath{= \{\spvier\}}}
\smaller{    } while x\mneq{}y
\smaller{    } do
\smaller{ 05.}   \allttmath{\{\pvijf\}} \smaller{ \allttmath{= \{\spvijf\}}}
\smaller{    }   if x>y
\smaller{    }   then
\smaller{ 06.}     \allttmath{\{\pzes\}} \smaller{ \allttmath{= \{\spzes\}}}
\smaller{ 07.}     \allttmath{\{\pzeven\}} \smaller{ \allttmath{= \{\spzeven\}}}
\smaller{    }     x:=x-y;
\smaller{ 08.}     \allttmath{\{\pvier\}} \smaller{ \allttmath{= \{\spvier\}}}
\smaller{    }   else
\smaller{ 09.}     \allttmath{\{\pacht\}} \smaller{ \allttmath{= \{\spacht\}}}
\smaller{ 10.}     \allttmath{\{\pnegen\}} \smaller{ \allttmath{= \{\spnegen\}}}
\smaller{ 11.}     \allttmath{\{\ptien\}} \smaller{ \allttmath{= \{\sptien\}}}
\smaller{    }     y:=y-x;
\smaller{ 12.}     \allttmath{\{\pvier\}} \smaller{ \allttmath{= \{\spvier\}}}
\smaller{ 13.}   \allttmath{\{\pvier\}} \smaller{ \allttmath{= \{\spvier\}}}
\smaller{ 14.} \allttmath{\{\pelf\}} \smaller{ \allttmath{= \{\spelf\}}}
\smaller{ 15.} \allttmath{\{\ptwaalf\}} \smaller{ \allttmath{= \{\sptwaalf\}}}
\smaller{ 16.} \allttmath{\{\pdertien\}} \smaller{ \allttmath{= \{\spdertien\}}}
\smaller{    } g:=x;
\smaller{ 17.} \allttmath{\{\pveertien\}} \smaller{ \allttmath{= \{\spveertien\}}}
\smaller{ 18.} \allttmath{\{\pvijftien\}} \smaller{ \allttmath{= \{\spvijftien\}}}
\end{alltt}
\end{varwidth}
\caption{Euclid's algorithm proved by an annotated program}
\label{fig:euclidprog}
\end{center}
\end{figure}

We complete this example by explaining all the steps in the proof
of Figure~\ref{fig:euclidprog}, which implicitly also explains
all the steps in the proof of Figure~\ref{fig:euclidtree}.
\begin{enumerate}
\item \verb+01-->02+: [cons$_p$], elementary mathematics (EM).
\item \verb+02-->03+: [ass$_p$], $\sptwee = \spdrie\pass{x}{m}$.
\item \verb+03-->04+: [ass$_p$], $\spdrie = \spvier\pass{y}{n}$.
\item \verb+04-->14+: [while$_p$], based upon \verb+04+, \verb+05+, \verb+13+ and \verb+14+. Note that $\spvier$ is the loop invariant, $\spvijf = x\neq y \land \spvier$ and $\spelf = \neg(x\neq y) \land \spvier$.
\item \verb+05-->13+: [if$_p$], based upon \verb+05+, \verb+06+, \verb+08+, \verb+09+, \verb+12+ and \verb+13+.
Note that $\spzes = \left.x>y\right. \land \spvijf$, 
and $\spacht = \neg(x>y) \land \spvijf$. 
\item \verb+06-->07+: [cons$_p$], based upon (\ref{eqn:euclidhint}).
\item \verb+07-->08+: [ass$_p$], $\spzeven = \spvier\pass{x}{x-y}$.
\item \verb+09-->10+: [cons$_p$], EM.
\item \verb+10-->11+: [cons$_p$], based upon (\ref{eqn:euclidhint}).
\item \verb+11-->12+: [ass$_p$], $\sptien = \spvier\pass{y}{y-x}$.
\item \verb+14-->15+: [cons$_p$], EM.
\item \verb+15-->16+: [cons$_p$], $x=y \wedge x>0 \wedge y>0$ implies $\mgcd(x,y)=x$.
\item \verb+16-->17+: [ass$_p$], $\spdertien = \spveertien\pass{g}{x}$.
\item \verb+17-->18+: [cons$_p$], EM.
\end{enumerate}

\bibliographystyle{plain}
\bibliography{sc}
\end{document}
