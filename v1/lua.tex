\documentclass{article}
\usepackage{geometry}
 \geometry{left = 3cm, right = 3cm, top=3cm, foot = 3cm, head = 2cm}
\usepackage[utf8]{inputenc}
\usepackage{syntax}
\usepackage{listings}

\title{Models of Calculation Project \\
         The Language Lua}
\author{Serena Rietbergen, Frank Gerlings, Lars Jellema}
\date{April 2016}

\begin{document}

\maketitle
\newpage

\tableofcontents
\newpage

\section{Introduction}

In this article, we're going to define axiomatic semantics for concurrency in
Lua. We're going to start by defining the basics of Lua, including some basic
data types which we'll use in our examples. Then follows a description of
coroutines, which is the built-in concurrency tool of Lua. Coroutines in Lua are
collaborative, which means a coroutine must explicitly yield its cpu to other
coroutines. Only one coroutine ever runs at any time, which means coroutines do
not offer parallelism. They still offer a useful abstraction though. As While
doesn't offer any kind of concurrency, we believe Lua is sufficiently special.

% TODO: Explain why coroutines are a useful abstraction

\subsection{Example code}

\lstinputlisting{example.lua}

%  Neem in de inleiding ook alvast een voorbeeldprogramma op waarin iets interessants gebeurt. Zo'n programma helpt om duidelijk te krijgen hoe programma's in jullie taal er uitzien.

\section{Description of the syntax}
% Dat kan natuurlijk al een complete grammatica zijn, maar op dit moment is het voldoende om hier te beschrijven welke constructies uitgewerkt gaan worden. In het bijzonder geef je hier aan of je de hele taal gaat beschrijven of slechts een deel er van.

The following section describes the grammar of the subset of Lua that we'll be
describing. Numerous Lua constructs have been left out for the sake of
simplicity, like multiple assignment for example. Whitespace has also been left
out of the grammar, as have the optional semicolons.

\newpage

\begin{grammar}

	<block> ::= <stmt> <block> | `break' | `'

	<expr> ::= <bool> | <number> | <string> | <table> | <func> | <nil>

	<nil> ::= `nil'

	<bool> ::= `true' | `false'

	<stmt> ::= <do> | <if> | <while> | <repeat> | <num-for> | <iter-for> |
	<assign> | <return>

	<assign> ::= <var> `=' <expr>

	<return> ::= `return' <expr>

	<do> ::= `do' <block> `end'

	<if> ::= `if' <expr> `then' <block> <else-ifs> <else> `end'

	<else-ifs> ::= `elseif' <expr> `then' <block> <else-ifs> | `'

	<else> ::= `else' <block> | `'

	<while> ::= `while' <expr> `do' <block> `end'

	<repeat> ::= `repeat' <block> `until' <expr>

	<num-for> ::= `for' <var> `=' <expr> `,' <expr> <num-for-step> `do' <block>
	`end'

	<num-for-step> ::= `,' <expr> | `'

	<iter-for> ::= `for' <var> `,' <var> `in' <expr> `do' <block> `end'

\end{grammar}

This grammar is still incomplete.

% TODO: Fill in <var>, <expr>, <bool>, <number>, <string>, <func-def>, <call>,
% <table>, <co-create>, <co-wrap>, <co-resume>, <co-yield>

\subsection{Expected Problems}
% Verwacht je nu al problemen bij de beschrijving van de syntax, benoem die hier dan al.

\section{Description of the semantics}
% Natuurlijk hoef je hier nog geen complete lijst met semantiekregels te geven, maar je moet al wel een idee hebben hoe je denkt te gaan werken.
\subsection{Approach}
Axiomatisch. Makkelijker voor stelling te bewijzen, ipv een stuk code te simuleren
%Ga je voor ns, sos of nog heel iets anders? En waarom?
\subsection{Concepts}
% Probeer ook vast iets te zeggen over de concepten die je nodig hebt:
Core: tables, metatables\\
Expansion: coroutines, yielding and resuming, pipes and filters  
\subsubsection{States}
%wat zijn je toestanden (als je toestanden gebruikt),
We are going to use Axiomatic Semantics and therefore we do not have states. We use pre- and postconditions instead.
\subsubsection{Transitions}
% wat zijn je transities,
Our transitions are of the form:\\
${P}S{Q}$\\
\subsubsection{Types}
% welke types spelen een rol, etcetera.

\subsection{Expected Problems}
% Verwacht je nu al problemen bij de beschrijving van de semantiek, benoem die hier dan al.

\section{Analysis}
\subsection{Example Code}
\subsection{Elaboration}
%Heb je bijvoorbeeld al een mooi stuk voorbeeldcode waarvan je uiteindelijk wil laten zien dat het precies doet wat in je semantiekregels hebt vastgelegd, geef dat voorbeeld dan reeds expliciet aan. Dit mag natuurlijk het voorbeeld uit de inleiding zijn, maar het hoeft niet.

\newpage
\appendix
\section{Planning}
% the \\ insures the section title is centered below the phrase: AppendixA



\end{document}
