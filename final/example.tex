\lstset{%
	language = [5.0]Lua,
	basicstyle = \ttfamily,
	tabsize = 4,
	keywordstyle = \bfseries,
}

\lstinputlisting[language={[5.0]Lua}]{code.lua}
In this example, a coroutine will be demonstrated. This coroutine will remember a pointer and, based on it's input, will increment or decrement this pointer.
\\ \texttt{co = coroutine.wrap(indec)} only creates the coroutine, it does not
start it. When \texttt{it0=co(false)} is called, the coroutine starts with
\texttt{x=false}. The coroutine then enters the while-loop, and because
\texttt{x=false} the program only executes the line
\texttt{pointer=pointer+delta} which gives the variable \texttt{pointer} the
value 1. When \texttt{coroutine.yield(pointer)} is executed, the coroutine
yields and returns the value from \texttt{pointer}. This gives the varablie
\texttt{it0} the value 1. Then \texttt{it1 = co(false)} will be executed. This
will resume the coroutine with \texttt{x=false}. The coroutine resumes where it
left off, which is in the while-loop. Because \texttt{x} is \texttt{false} the
coroutine will again only execute \texttt{pointer=pointer+delta}, which will
return the coroutine with the value of \texttt{pointer}, which is 2. This will
lead to \texttt{it1=2}. Then the coroutine is resumed with \texttt{true} in the
while-loop. Because it is resumed with \texttt{true}, \texttt{delta = delta *
(-1)} will be executed, giving \texttt{delta} the value of \texttt{-1}. After
this, \texttt{pointer=pointer + delta} will be executed. Lastly, the coroutine
will yield with \texttt{pointer=1}, giving \texttt{it2} the value of \texttt{1}.
\\ In the chapters 2 and 3 we will create syntactical and semantical rules, so that we can prove the correctness of this program in chapter 4.